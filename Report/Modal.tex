\chapter{Modal analysis}\label{chap:modalanalysis}
When external forces act on a multi degree-of-freedom system, the system
undergoes forced vibration.
For a system with three coordinates or degrees of freedom, the governing
equation of motion are a set of three coupled ordinary differential equations of
second second order.
The equations of motion of a multi degree-of-freedom system under external force
are given by:
\begin{equation}
  \label{eq:modal}
  [\mathbf{M}] \ddot{\vec{x}} + [\mathbf{K}] \vec{x} = \vec{\mathbf{F}}
\end{equation}
The equation \eqref{eq:modal} represents the system in the version without
damping, where \(\vec{\mathbf{F}}\) is the vector of the external force.
\section{Forced vibration of undamped system}\label{sec:eigen}
To solve the equation \eqref{eq:modal} by modal analysis, it is necessary first
to solve the eigenvalue problem and find the natural frequencies \(\omega_1,
\dots, \omega_n\) as reported below:
\(\omega_{1} = 8.27843\) [\si{\radian\per\second}],
\(\omega_{2} = 27.40935\) [\si{\radian\per\second}],
\(\omega_{3} = 41.87035\) [\si{\radian\per\second}];
and the corresponding normal modes:
\begin{equation}
	\label{eq:eiegnmodefree}
 	[\mathbf{U}] = \begin{bmatrix*}[r]
 		1.00000 & 1.00000 & 1.00000 \\
		0.86540 &-0.47555 &-2.44325 \\
		0.61945 &-1.28034 & 2.15533 \\
 	\end{bmatrix*}
\end{equation}
\section{Forced vibration of viscous damped system}\label{sec:viscoussystem}
Modal analysis, as presented in previous section \ref{sec:eigen}, applies only
to undamped systems.
In many cases, the influence of damping upon the response of a vibratory system
is minor and can be disregarded. However, it must be considered if the response
of the system is required for a relatively long period of time compared to the
natural periods of the system. Further, if the frequency of excitation is at or
near one of the natural frequencies of the system, damping is of primary
importance and must be taken into account. In general, since the effects are not
known in advance, damping must be considered in the vibration analysis of any
system, thus using the equation of motion:
\begin{equation}\label{eq:modaldamping}
  [\mathbf{M}] \ddot{\vec{x}} + [\mathbf{C}] \dot{\vec{x}} +[\mathbf{K}] \vec{x}
   = \vec{\mathbf{F}}
\end{equation}
%
Then we can consider for which the damping matrix can be expressed as a linear
combination of the mass and stiffness matrices:
\begin{equation}
\label{eq:moadlpropdamping}
	[C] = \alpha \cdot [\mathbf{M}] + \beta \cdot [\mathbf{K}]
\end{equation}
where \(\alpha\) and \(\beta\) are constant. By substituting in
\eqref{eq:modaldamping}, we obtain:
\begin{equation}\label{eq:modaldampingcomplete}
	[\mathbf{M}] \ddot{\vec{x}} +
	[\alpha [\mathbf{M}] + \beta[\mathbf{K}]] \dot{\vec{x}} +
	[\mathbf{K}] \vec{x}= \vec{\mathbf{F}}
\end{equation}
By expressing the solution vector \(\vec{x}\) as linear combination of the
natural mode of the undamped system. Substituting \(x(t) = [\mathbf{U}]q(t)\)
in the equation \eqref{eq:modaldampingcomplete} and pre-multiplying by 
\([\mathbf{U}]^{\top}\) we obtain:
\begin{equation}\label{eq:modaldampingexpansed}
	[\mathbf{U}]^{\top}[\mathbf{M}][\mathbf{U}] \ddot{\vec{q}} +
	[\alpha[\mathbf{U}]^{\top}[\mathbf{M}][\mathbf{U}] +
   \beta[\mathbf{U}]^{\top}[\mathbf{K}]][\mathbf{U}] \dot{\vec{q}} +
	[\mathbf{U}]^{\top}[\mathbf{K}][\mathbf{U}]^{\top} \vec{q} =
  [\mathbf{U}]^{\top}\vec{\mathbf{F}}
\end{equation}
%
If the eigenvectors \(\vec{\mathbf{U}}^{(j)}\) are normalized we can rewrite in 
decoupled form:
\begin{equation}\label{eq:decupled}
  \ddot{q}_{i}(t) + (\alpha + \omega_{i}^2\beta)\dot{q}_{i}(t) +
  \omega_{i}^{2}q(t) = Q_{i}(t)
\end{equation}
where: \(\omega_{1} = 8.27799\) [\si{\radian\per\second}],
\(\omega_{2} = 27.40863\) [\si{\radian\per\second}],
\(\omega_{3} = 41.91576\) [\si{\radian\per\second}] are
the natural frequencies of the undamped system.
The eigenvector are reported in \eqref{eq:eiegnmodeprop}.
\begin{equation}\label{eq:eiegnmodeprop}
	[\mathbf{U}] = \begin{bmatrix*}[r]
		1.00000 & 1.00000 & 1.00000 \\
		0.86499 &-0.48006 &-2.46147 \\
		0.61907 &-1.28528 & 2.16954 \\
 	\end{bmatrix*}
\end{equation}
In the equation \eqref{eq:decupled} substituting the relationship with 
\(\alpha + \omega^{2}_{i}\beta = 2\,\zeta_{i}\,\omega_{i}\), where
\(\zeta_{i}\) is called \emph{modal damping ratio} for the \(i_\text{th}\) normal
mode, equation \eqref{eq:decupled} we obtain:
\begin{equation}\label{eq:decupled2}
  \ddot{q}_{i}(t) + 2\zeta_{i}\omega_{i}\dot{q}_{i}(t) +
  \omega_{i}^{2}q(t) = Q_{i}(t)
\end{equation}
It can be seen that each of then equations represented by this expression is
uncoupled from all of the others.
Hence we can find the response of the \(i_\text{th}\) mode in the same manner
as that of a viscously damped single-degree-of-freedom system.
%
The identification of the sources and magnitude of damping is difficult in most
practical problems.
The condition given by equation \eqref{eq:moadlpropdamping} is sufficient but not
necessary for the existence of normal modes in damped systems.
The necessary condition is that the transformation that diagonalizes the damping
matrix also uncouples the coupled equations of motion.
This condition is less restrictive than equation \eqref{eq:moadlpropdamping}
and covers more possibilities.
In the general case of damping, the damping matrix cannot be diagonalized
simultaneously with the mass and stiffness matrices.
A common procedure for finding the solution of the eigenvalue problem of a
damped system involves the transformation of the coupled second-order
equations of motion into \(2n\) uncoupled first-order equation.
%
\section{Rayleigh quotient method}
\label{sec:ray}
Rayleigh's method can be extended to find the approximate value of the fundamental
natural frequency of a discrete system.
It is possible derive an expression for the approximate value of the first
natural frequency of a multi degree-of-freedom system according to Rayleigh's
method.
The kinetic and potential energies of an \(n\)-degree-of-freedom discrete system
can be expressed as:
\begin{align}\label{eq:energyequation}
  T &= \frac{1}{2}\dot{\vec{x}}^{\top}[\mathbf{M}]\dot{\vec{x}}\\[1em]
  V &= \frac{1}{2}\dot{\vec{x}}^{\top}[\mathbf{K}]\dot{\vec{x}}
\end{align}
To find the natural frequencies, we assume harmonic motion to be
\(\vec{x} = \vec{\mathbf{X}} \cos(\omega t)\), where \(\vec{\mathbf{X}}\) 
denotes the vector of amplitudes and \(\omega\) represents the natural frequency
of vibration.
If the system is conservative, the maximum kinetic energy is equal to the
maximum potential energy. By equating \(T\) and \(V\) obtain:
\begin{equation}\label{eq:quotient}
  \omega^2 = \frac{\dot{\vec{x}}^{\top}[\mathbf{K}]\dot{\vec{x}}}
  {\dot{\vec{x}}^{\top}[\mathbf{M}]\dot{\vec{x}}}
\end{equation}
%
The right-hand side of \eqref{eq:quotient} is known as \emph{Rayleigh's quotient}
and is denoted as \(R(\vec{X})\), then the Rayleigh's quotient can be expressed
as:
\begin{align}
  \omega^{2} =& R(\vec{X}) =
  \frac{
          c_{1}^{2}\,\omega_{2}^{2}\,\dot{\vec{x}}^{\top}\,[\mathbf{M}]\,\dot{\vec{x}} +
          c_{2}^{2}\,\omega_{2}^{2}\,\dot{\vec{x}}^{\top}\,[\mathbf{M}]\,\dot{\vec{x}} +
          \dots}
       {
          c_{1}^{2}\,\dot{\vec{x}}^{\top}\,[\mathbf{M}]\,\dot{\vec{x}} +
          c_{2}^{2}\,\dot{\vec{x}}^{\top}\,[\mathbf{M}]\,\dot{\vec{x}} + \dots
        }\\[1.5em]
      &  R(\vec{X}) =
      \frac{c_{1}^{2}\,\omega_{2}^{2} + c_{2}^{2}\,\omega_{2}^{2} + \dots}
           {c_{1}^{2} + c_{2}^{2} + \dots}
\end{align}
The results obtained with regard to the natural frequencies are as follows:
\(\omega_{1} = 8.27843\) [\si{\radian\per\second}],
\(\omega_{2} = 27.40935\) [\si{\radian\per\second}],
\(\omega_{3} = 41.87035\) [\si{\radian\per\second}].
Instead the values obtained for the mode shapes are:
\begin{equation}
  \label{eq:raymodefree}
  [\mathbf{U}] = \begin{bmatrix*}[r]
	1.00000 & 1.00000 & 1.00000 \\
	0.86540 &-0.47555 &-2.44325 \\
	0.61945 &-1.28034 & 2.15533 \\
  \end{bmatrix*}
\end{equation}
The results obtained for the proportionally damped system are the natural 
frequencies:
\(\omega_{1} =  8.27799\) [\si{\radian\per\second}],
\(\omega_{2} = 27.40863\) [\si{\radian\per\second}],
\(\omega_{3} = 41.91576\) [\si{\radian\per\second}].
So the values obtained for the mode shapes are
\begin{equation}\label{eq:raymodeprop}
  [\mathbf{U}] = 
	\begin{bmatrix*}[r]
  		1.00000 & 1.00000 & 1.00000 \\
		0.86499 &-0.48006 &-2.46147 \\
		0.61907 &-1.28528 & 2.16954 \\
	\end{bmatrix*}
\end{equation}
Rayleigh's quotient has a stationary value in the neighbourhood of an eigenvector. 
The stationary value is actually a minimum value in the neighbourhood of the 
fundamental mode.
Which means that Rayleigh s quotient is never higher than the highest eigenvalue. 
Thus Rayleigh's quotient provides an upper bound for \(\omega_{1}^{2}\) lower bound 
for \(\omega_{n}^{2}\).

\section{Matrix Iteration Method}\label{sec:mim}
the matrix iteration method assumes that the natural frequencies are distinct
and well separated such that \(\omega_1 < \omega_2 < \dots < \omega_n\).
The iteration starting by selecting a trial vector \(\vec{\mathbf{X}}_{1}\),
which is then premultiplied by the dynamical matrix \([D]\).
The resulting column vector is then normalized, usually by making one of its
components equal to unity.
The normalized column vector is premultiplied by \([D]\) to obtain a third
column vector, which is normalized in the same way as before and become still
another trial column vector.
The process is repeated until the successive column vectors converge to a common
vector: the fundamental eigenvector.
According to the expansion theorem, any arbitrary \(n\)-dimensional vector
\(\vec{\mathbf{X}}_{1}\) can be expressed as linear combination of the \(n\)
orthogonal eigenvectors of the system \(\vec{\mathbf{X}}^{(i)}\).
\begin{equation}\label{eq:expasniontheorem}
	\sum_{i = 1}^{n} = c_{i}\,\vec{\mathbf{X}}^{(n)}
\end{equation}
where \(c_{i}\) are unknown constant number to be determined.
As view before, it is possible premultiply \(\vec{\mathbf{X}}_{1}\) by matrix
\([D]\) obtaining:
\begin{equation}\label{eq:equationmultiply}
	[D] \vec{\mathbf{X}}_{1} = 	c_{1}\, [D] \,\vec{\mathbf{X}}_{(1)} +
								c_{2}\, [D] \,\vec{\mathbf{X}}_{(2)} + \dots +
  							c_{n}\, [D] \,\vec{\mathbf{X}}_{(n)}\\
\end{equation}
In according with the equation \(\lambda\,[I]\,\vec{X} = [D]\,\vec{X}\), then
we obtain
\(\vec{X}^{(n)} = \lambda\,[I]\,\vec{X}^{(n)} = \frac{1}{\omega^{2}_{n}}
\vec{X}^{(n)}\), thus after having substituted it in the equation
\eqref{eq:equationmultiply} one comes to the:
\begin{align}\label{eq:equationsubs}
  	[D]\,\vec{\mathbf{X}}_{1} &= \vec{\mathbf{X}}_{2}\\[0.75em]
      						&= \frac{c_{1}}{\omega^{2}_{1}}\,\vec{\mathbf{X}}^{(1)} +
		\frac{c_{2}}{\omega^{2}_{2}}\,\vec{\mathbf{X}}^{(2)} + \dots +
		\frac{c_{n}}{\omega^{2}_{n}}\,\vec{\mathbf{X}}^{(n)}
\end{align}
By repeating the process we obtain, after \(r_{th}\) iteration,
\begin{align}\label{eq:equationrth}
  	[D]\,\vec{\mathbf{X}}_{r} &= \vec{\mathbf{X}}_{r+1}\\[0.75em]
      						&= \frac{c_{1}}{\omega^{2r}_{1}}\,\vec{\mathbf{X}}^{(1)} +
		\frac{c_{2}}{\omega^{2r}_{2}}\,\vec{\mathbf{X}}^{(2)} + \dots +
		\frac{c_{n}}{\omega^{2r}_{n}}\,\vec{\mathbf{X}}^{(n)}
\end{align}
Since the natural frequencies are are assumed \(\omega_1 < \omega_2 < \dots <
\omega_n\), a sufficiently large value of \(r\) yields \(\frac{1}{\omega_1^{2r}}
>> \frac{1}{\omega_2^{2r}} >> \dots >> \frac{1}{\omega_n^{2r}}\). Thus the first
term in right-hand side of equation \eqref{eq:equationrth} becomes the only
significant one.
Which means that the \((r+1)_{th}\) trial vector becomes identical to the
fundamental modal vector to within a multiplicative constant
\(\vec{\mathbf{X}}_{r} = \frac{c_1}{\omega^{2(r-1)}_{1}}\vec{\mathbf{X}}^{(1)}\).
Then the fundamental natural frequency \(\omega_1\) can be found by taking the
ratio of any two corresponding components in the vectors \(\vec{\mathbf{X}}_{r}\)
and \(\vec{\mathbf{X}}_{r+1}\):
\begin{equation}
  \omega_1^{2} \simeq \frac{\vec{\mathbf{X}}_{n,r}}{\vec{\mathbf{X}}_{n,r+1}}
\end{equation}
where \(\vec{\mathbf{X}}_{n,r}\) and \(\vec{\mathbf{X}}_{n,r+1}\) are the
\(n_{th}\) elements of vector \(\vec{\mathbf{X}}_{r}\) and
\(\vec{\mathbf{X}}_{r+1}\), respectively.
\subsection{Intermediate natural frequencies}
Once the first natural frequency \(omega_1\) and the corresponding eigenvector
\(\vec{\mathbf{X}}^{(1)}\) are determined, it is possible proceed to find the
higher natural frequencies and the corresponding mode shapes.
To find the eigenvector \(\vec{\mathbf{X}}^{(i)}\), the previous eigenvector
\(\vec{\mathbf{X}}^{(i-1)}\) is normalized with respect to the mass matrix such
that \(\vec{\mathbf{X}}^{(i-1)^\top}\,[m]\,\vec{\mathbf{X}}^{(i-1)} = 1\).
The deflated matrix \([D_i]\) is the constructed as
\begin{equation}
  [D_i] =
  [D_i] - \lambda_{i-1}\vec{\mathbf{X}}^{(i-1)}\,\vec{\mathbf{X}}^{(i-1)^\top}[m]
\end{equation}
where \([D_1] = [D]\).
Once \([D_i]\) is constructed the iterative scheme
\begin{equation}
  \vec{\mathbf{X}}^{(r+1)} = [D_i]\,\vec{\mathbf{X}}^{(r)}
\end{equation}
is used.
\subsection{Result}\label{ssec:resultmim}
The whole procedure is performed for the case of free damping where the natural
frequencies are: \(\omega_{1} = 8.27843\) \si{\radian\per\second}, \(\omega_{2}
= 27.41224\) \si{\radian\per\second}, \(\omega_{3} = 41.85486\)
\si{\radian\per\second}.
The mode shapes results are shown in \eqref{eq:mimmodefree}.
\begin{equation}\label{eq:mimmodefree}
 \begin{bmatrix*}[r]
	1.00000 & 1.00000 & 1.00000 \\
	0.86540 & 0.21230 &-0.85499 \\
	0.61945 &-0.30643 &-0.30390 \\
	 \end{bmatrix*}
\end{equation}
%
In the case of proportional damping the natural frequencies are:
\(\omega_{1} = 8.27799\) \si{\radian\per\second}, \(\omega_{2} = 27.40863\)
\si{\radian\per\second}, \(\omega_{3} = 41.91576\) \si{\radian\per\second}.
The mode shapes results are shown in \eqref{eq:mimmodeprop}.
 \begin{equation}\label{eq:mimmodeprop}
 \begin{bmatrix*}[r]
	1.00000 & 1.00000 & 1.00000 \\
	0.86499 &-0.48006 &-2.46147 \\
	0.61907 &-1.28528 & 2.16954 \\
	 \end{bmatrix*}
\end{equation}
\subsection{Observation}\label{ssec:observationmim}
Although it is theoretically necessary to have \(r\rightarrow\infty\) for the 
convergence of the method, in practice only a finite number of iterations
suffices to obtain a reasonably good estimate of \(\omega_{1}\).
The actual number of iterations necessary to find the value of \(\omega_1\) to
within a desired degree of accuracy depends on how closely the arbitrary trial
vector \(\vec{\mathbf{X}}_{1}\) resembles the mode and how well \(\omega_1\)
and \(\omega_2\) are separated.
The required number of iterations is less if \(\omega_2\) is very large compared
to \(\omega_{1}\).
The method has a distinct advantage in that any computational errors made do not
yield incorrect results. Any error made in premultiplying
\(\vec{\mathbf{X}}_{i}\) by \([D]\) results in a vector other than the desired
one, \(\vec{\mathbf{X}}_{i+1}\).
But this wrong vector can be considered as a new trial vector.
This may delay the convergence but does not produce wrong results.


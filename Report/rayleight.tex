\section{Rayleigh}
\label{sec:ray}
Rayleigh's method can be extended to find the approximate value of the fundamental
natural frequency of a discrete system.
It is possible derive an expression for the approximate value of the first
natural frequency of a multidegree-of-freedom system according to Rayleigh's
method.
The kinetic and potential energies of an \(n\)-degree-of-freedom discrete system
can be expressed as
\begin{align}\label{eq:energyequation}
  T &= \frac{1}{2}\dot{\vec{x}}^{\top}[\mathbf{M}]\dot{\vec{x}}\\[1em]
  V &= \frac{1}{2}\dot{\vec{x}}^{\top}[\mathbf{K}]\dot{\vec{x}}
\end{align}
%
To find the natural frequencies, we assume harmonic motion to be
\(\vec{x} = \vec{\mathbf{X}} \cos(\omega t)\) where \(\vec{\mathbf{X}}\) denotes
the vector of amplitudes and \(\omega\) represents the natural frequency of
vibration.
If the system is conservative, the maximum kinetic energy is equal to the
maximum potential energy. By equating \(T\) and \(V\) obtain:
\begin{equation}\label{eq:quotient}
  \omega^2 = \frac{\dot{\vec{x}}^{\top}[\mathbf{K}]\dot{\vec{x}}}
  {\dot{\vec{x}}^{\top}[\mathbf{M}]\dot{\vec{x}}}
\end{equation}
%
The right-hand side of \eqref{eq:quotient} is known as \emph{Rayleigh's quotient}
and is denoted as \(R(\vec{X})\), then the Rayleigh's quotient can be expressed
as:
\begin{align}
  \omega^{2} =& R(\vec{X}) =
  \frac{
          c_{1}^{2}\,\omega_{2}^{2}\,\dot{\vec{x}}^{\top}\,[\mathbf{M}]\,\dot{\vec{x}} +
          c_{2}^{2}\,\omega_{2}^{2}\,\dot{\vec{x}}^{\top}\,[\mathbf{M}]\,\dot{\vec{x}} + \dots}
       {
          c_{1}^{2}\,\dot{\vec{x}}^{\top}\,[\mathbf{M}]\,\dot{\vec{x}} +
          c_{2}^{2}\,\dot{\vec{x}}^{\top}\,[\mathbf{M}]\,\dot{\vec{x}} + \dots
        }\\[1.5em]
      &  R(\vec{X}) =
      \frac{c_{1}^{2}\,\omega_{2}^{2} + c_{2}^{2}\,\omega_{2}^{2} + \dots}
           {c_{1}^{2} + c_{2}^{2} + \dots}
\end{align}
The results obtained with regard to the natural frequencies are as follows:
\(\omega_{1} = 8.27843\) [\si{\radian\per\second}],
\(\omega_{2} = 27.40935\) [\si{\radian\per\second}],
\(\omega_{3} = 41.87035\) [\si{\radian\per\second}].
Instead the values obtained for the mode shapes are:
\begin{equation}
  \label{eq:raymodefree}
  [\mathbf{U}] = \begin{bmatrix*}[r]
	1.00000 & 1.00000 & 1.00000 \\
	0.86540 &-0.47555 &-2.44325 \\
	0.61945 &-1.28034 & 2.15533 \\
  \end{bmatrix*}
\end{equation}
%!
%(7.15)
% 7.3.1 Asstatedeahralisear,sRta(tXio)naryvaluewhenthearbitraryvectoisrintheneigh- X !!
%(r)
%Properties borhood of any eigenveTcotoprrXove.this, we express the arbitrary vectoinr terms
%ofRayleigh s Quotient
%X
% !(i)
% of the normal modes of the systaesm, X !     !(1)
%,
%+ c2X
%!(2)
%!T ! 2 !(1)T !(1) 2 !(2)T !(2)
%!(3) Á + c3X +
% Then
%X = c1X
%X [k]X = c1 X [k]X + c2X [k]X
%     2 !(3)T !(3) Á + c3X [k]X +
% (7.16) 2 !2!!T
%   Equation (7.14) can also be obtained from the relation [k]X = v [m]X. Premultiplying this equation by X and solving the resulting equation gives Eq. (7.14).
%
%                                        660 CHAPTERD7ETERMINATION OF NATURAL FREQUENCIES AND MODE SHAPES
%and
%!T ! 2 !(1)T !(1) 2 !(2)T !(2) X[m]X = c1X [m]X + c2X [m]X
%2!(3)T !(3)Á + c3X [m]X +
%(7.17)
%       !(i)T !(j) !(i)T !(j)
%as the cross terms of the form cicjX [k]X and cicjX [m]X , i Z j, are zero by the
%   orthogonality property. Using Eqs. (7.16) and (7.17) and the relation
%!(i)T !(i) 2 !(i)T !(i) X [k]X = viX [m]X
%the Rayleigh s quotient of Eq. (7.14) can be expressed as
%(7.18)
%(7.19)
%(7.20)
%    2 2 !(1)T !(1) 2 ! c1 v1X [m]X
%v =R(X)= 2 !(1)T !(1) c1X [m]X
%2 2 !(2)T + c2v2X
%!(2) [m]X
%+ Á
%Á
%    2 !(2)T !(2) + c2 X [m]X
%+
%    If the normal modes are normalized, this equation becomes
%2 ! c21v21 + c2v2 + Á v = R ( X ) = c 21 + c 2 2 + Á
%! !(r)
%dIfifXfers little from the eigenvtehcetocroefficiXewntil,l be much largercthan the
%remaining coaenfdfiEcqie.n(t7s.c20()icZanrb)e, written as i
%   222 ci22 crvr+cr a +c*vi
%i=1, 2, Á r ! iZr
%(7.21)
%(7.22)
%r
% R(X)=
%ci 2 c2r+c2r a +c*
%i=1, 2, Á r iZr
%  Since ci/cr   = ei V 1, wheries aeismall number Efoqr.a(l7l.21) giivZesr, !22
%R(X) = vr51 + 0(e )6 wrehperese0n(ets)an expression ionf the second oreder or higher. Equation (7.22)
%2
%indicates that if the arbitrary vectodrifXfers from the eigenvebctyora smaXll quantity
%!
%2
%of the first odridffeerr,s fRro(Xm)the eigenvalbuyea small quantityvof the second
%! !(r)
%  order. This means that Rayleighs quotient has a stationary value in the neighborhood of an eigenvector.
%r
%
%
%The stationary value is actually a minimum value in the neighborhood of the funda-
%!(1)
%mental mToodes,eXe thiin.s,Elqet. (7.21) and rwr=ite1
%2 ci 2 2 ! v1+ a ¢c vi
%i = 2, 3, Á 1
%7R.3AYLEIGH S METHOD 661
%    R ( X) =
%b1+ a ¢c r
% ci 2 i = 2, 3, Á 1
%M v 21 + a e 2i v 2i - v 21 a e 2i i = 2, 3, Á i = 2, 3, Á
%M v21 + a (v2i - v21)e2i i=2, 3, Á
%Since, infogreneral, v2i 7 v21 i = 2, 3, Á , Eq. (7.23) leads to !2
%(7.23)
%(7.24)
%R(X) Ú v1
%which shows that Rayleigh s quotient is never lower than the first eigenvalue. By proceed-
%ing in a similar manner, we can show that
%R(X)   vn (7.25)
%!2
%which means that Rayleigh s quotient is never higher than the highest eigenvalue. Thus
% 22 Rayleigh s quotient provides an upper bound faonrdva lower bound for v .
%7.3.2 Computation of the Fundamental Natural Frequency
%EXAMPLE7.2
%Equation (7.14) can be used to find an approximate value of the first natural frequenc(vy 1) ! !(1)
%Because Rayleighs quotient is stationary, remarkably good estimatescovafn be obtained !!1
%1n
% ofthesystem.Forthis,weselectatrialvectotXorrepresentthefirstnaturalmaondde X substitute it on the right-hand side of Eq. (7.14). This yields the approximate value of v21.
%2
%  (1) evenifthetrialvectdoervXiatesgreatlyfromthetruenaturalmOobdveiously,tXhe . !
%estimatedvalueofthefundamentalfrequenicsymorveaccurateifthetrialvector (X)
%!(1) Dunkerley s and other methods in Refs. [7.7 7.9].


%
%Proportional damping:
% frequencies
\(\omega_{1} =  8.27799\) [\si{\radian\per\second}],
\(\omega_{2} = 27.40863\) [\si{\radian\per\second}],
\(\omega_{3} = 41.91576\) [\si{\radian\per\second}]
% modes:
\begin{equation}\label{eq:raymodeprop}
  [\mathbf{U}] = \begin{bmatrix*}[r]
    1.00000 & 1.00000 & 1.00000 \\
	  0.86499 &-0.48006 &-2.46147 \\
	  0.61907 &-1.28528 & 2.16954 \\
	\end{bmatrix*}
\end{equation}
%%

\section{Matrix Iteration Method}\label{sec:mim}
the matrix iteration method assumes that the natural frequencies are distinct
and well separated such that \(\omega_1 < \omega_2 < \dots < \omega_n\).
The iteration starting by selecting a trial vector \(\vec{\mathbf{X}}_{1}\),
which is then premultiplied by the dynamical matrix \([D]\).
The resulting column vector is then normalized, usually by making one of its
components equal to unity.
The normalized column vector is premultiplied by \([D]\) to obtain a third
column vector, which is normalized in the same way as before and become still
another trial column vector.
The process is repeated until the successive column vectors converge to a common
vector: the fundamental eigenvector.
%MODAL ANALYSIS - Matrix Iteration Method:
%
The whole procedure is performed for the case of free damping where the natural
frequencies are: \(\omega_{1} = 8.27843\) \si{\radian\per\second}, \(\omega_{2}
= 27.41224\) \si{\radian\per\second}, \(\omega_{3} = 41.85486\)
\si{\radian\per\second}.
The mode shapes results are shown in \eqref{eq:mimmodefree}.
 \begin{equation}
\label{eq:mimmodefree}
 \begin{bmatrix*}[r]
	1.00000 & 1.00000 & 1.00000 \\
	0.86540 & 0.21230 &-0.85499 \\
	0.61945 &-0.30643 &-0.30390 \\
	 \end{bmatrix*}
\end{equation}
%
In the case of proportional damping the natural frequencies are:
\(\omega_{1} = 8.27799\) \si{\radian\per\second}, \(\omega_{2} = 27.40863\)
\si{\radian\per\second}, \(\omega_{3} = 41.91576\) \si{\radian\per\second}.
The mode shapes results are shown in \eqref{eq:mimmodeprop}.
 \begin{equation}\label{eq:mimmodeprop}
 \begin{bmatrix*}[r]
	1.00000 & 1.00000 & 1.00000 \\
	0.86499 &-0.48006 &-2.46147 \\
	0.61907 &-1.28528 & 2.16954 \\
	 \end{bmatrix*}
\end{equation}

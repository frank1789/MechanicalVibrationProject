\section{Rayleigh quotient method}
\label{sec:ray}
Rayleigh's method can be extended to find the approximate value of the fundamental
natural frequency of a discrete system.
It is possible derive an expression for the approximate value of the first
natural frequency of a multidegree-of-freedom system according to Rayleigh's
method.
The kinetic and potential energies of an \(n\)-degree-of-freedom discrete system
can be expressed as
\begin{align}\label{eq:energyequation}
  T &= \frac{1}{2}\dot{\vec{x}}^{\top}[\mathbf{M}]\dot{\vec{x}}\\[1em]
  V &= \frac{1}{2}\dot{\vec{x}}^{\top}[\mathbf{K}]\dot{\vec{x}}
\end{align}
To find the natural frequencies, we assume harmonic motion to be
\(\vec{x} = \vec{\mathbf{X}} \cos(\omega t)\), where \(\vec{\mathbf{X}}\) 
denotes the vector of amplitudes and \(\omega\) represents the natural frequency
of vibration.
If the system is conservative, the maximum kinetic energy is equal to the
maximum potential energy. By equating \(T\) and \(V\) obtain:
\begin{equation}\label{eq:quotient}
  \omega^2 = \frac{\dot{\vec{x}}^{\top}[\mathbf{K}]\dot{\vec{x}}}
  {\dot{\vec{x}}^{\top}[\mathbf{M}]\dot{\vec{x}}}
\end{equation}
%
The right-hand side of \eqref{eq:quotient} is known as \emph{Rayleigh's quotient}
and is denoted as \(R(\vec{X})\), then the Rayleigh's quotient can be expressed
as:
\begin{align}
  \omega^{2} =& R(\vec{X}) =
  \frac{
          c_{1}^{2}\,\omega_{2}^{2}\,\dot{\vec{x}}^{\top}\,[\mathbf{M}]\,\dot{\vec{x}} +
          c_{2}^{2}\,\omega_{2}^{2}\,\dot{\vec{x}}^{\top}\,[\mathbf{M}]\,\dot{\vec{x}} +
          \dots}
       {
          c_{1}^{2}\,\dot{\vec{x}}^{\top}\,[\mathbf{M}]\,\dot{\vec{x}} +
          c_{2}^{2}\,\dot{\vec{x}}^{\top}\,[\mathbf{M}]\,\dot{\vec{x}} + \dots
        }\\[1.5em]
      &  R(\vec{X}) =
      \frac{c_{1}^{2}\,\omega_{2}^{2} + c_{2}^{2}\,\omega_{2}^{2} + \dots}
           {c_{1}^{2} + c_{2}^{2} + \dots}
\end{align}
The results obtained with regard to the natural frequencies are as follows:
\(\omega_{1} = 8.27843\) [\si{\radian\per\second}],
\(\omega_{2} = 27.40935\) [\si{\radian\per\second}],
\(\omega_{3} = 41.87035\) [\si{\radian\per\second}].
Instead the values obtained for the mode shapes are:
\begin{equation}
  \label{eq:raymodefree}
  [\mathbf{U}] = \begin{bmatrix*}[r]
	1.00000 & 1.00000 & 1.00000 \\
	0.86540 &-0.47555 &-2.44325 \\
	0.61945 &-1.28034 & 2.15533 \\
  \end{bmatrix*}
\end{equation}
The results obtained for the proportionally damped system are the natural 
frequencies:
\(\omega_{1} =  8.27799\) [\si{\radian\per\second}],
\(\omega_{2} = 27.40863\) [\si{\radian\per\second}],
\(\omega_{3} = 41.91576\) [\si{\radian\per\second}].
So the values obtained for the mode shapes are
\begin{equation}\label{eq:raymodeprop}
  [\mathbf{U}] = 
	\begin{bmatrix*}[r]
  		1.00000 & 1.00000 & 1.00000 \\
		0.86499 &-0.48006 &-2.46147 \\
		0.61907 &-1.28528 & 2.16954 \\
	\end{bmatrix*}
\end{equation}
Rayleigh's quotient has a stationary value in the neighbourhood of an eigenvector. 
The stationary value is actually a minimum value in the neighbourhood of the 
fundamental mode.
Which means that Rayleigh s quotient is never higher than the highest eigenvalue. 
Thus Rayleigh's quotient provides an upper bound for \(\omega_{1}^{2}\) lower bound 
for \(\omega_{n}^{2}\).
